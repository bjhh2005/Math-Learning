\documentclass[../main.tex]{subfiles}

\begin{document}

\section{Preliminaries}
\subsection{Set and Collections}
\begin{purple}
\begin{definition}
    Let $\mathcal{D}$ be a collection of sets. The union of $\mathcal{C}$, denoted by $\bigcup \mathcal{C}$, is the set defined by
    $$
    \bigcup\mathcal{C}=\{x|x\in S \text{ for some } S\in \mathcal{C}\}.
    $$

    If $\mathcal{C}$ is a non-empty collection, its intersection is the set $\bigcap\mathcal{C}$ given by 
    $$
    \bigcap\mathcal{C} = \{x|x\in S \text{ for every } S \in \mathcal{C}\}.
    $$
\end{definition}
\end{purple}

\begin{green}
Here, the generalized union and generalized intersection of sets simply involve performing the union or intersection operation on each element within the set.
\end{green}

Note that if $\mathcal{C}$ and $\mathcal{D}$ are two collections such that $\mathcal{C}\subseteq\mathcal{D}$, then
$$
\bigcup \mathcal{C}\subseteq \bigcup\mathcal{D} \text{ and }
\bigcap \mathcal{D} \subseteq \bigcap \mathcal{C}
$$

Within the framework of collections of subsets of a given set $\mathcal{S}$, we extend the previous definition by taking $\bigcap \varnothing = S$ for the empty collection of subsets of $\mathcal{S}$.

\begin{green}
    This is consistent with the fact that $\varnothing \subseteq \mathcal{C}$ implies $\bigcap \mathcal{C} \subseteq \mathcal{S}$. 
\end{green}

\begin{purple}
\begin{definition}
    The symmetric difference of sets denoted by $\oplus$ is defined by $U \oplus V = (U - V) \cup (V - U)$ for all sets $U, V$.
\end{definition}
\end{purple}

The symmetric difference operation is easily shown to satisfy symmetry and associativity, and also $U\oplus U=\varnothing$.

Next, we will prove the associativity.

\begin{proof}

    \begin{align*}
    \text{LEFT} 
    &= ((U - V) \cup (V - U)) \oplus T\\
    &= (((U - V) \cup (V - U)) - T) \cup (T - ((U - V) \cup (V - U)))\\
    &= ((U\bar V \cup V\bar U)\cap \bar T) \cup (T\cap (\bar U\bar V \cup UV))\\
    &= \sum_{i=1,2,4,7}m_i\\
    \\
    \text{RIGHT}
    &=\text{LEFT}\\
\end{align*}

\end{proof}

However, the above method is rather cumbersome. We can instead adopt the characteristic function approach for the proof. Here, we first supplement the relevant concepts of characteristic functions.

\begin{purple}
\begin{definition}
For any set $A$, the function
$$
\chi_A(x)=\begin{cases}
1, & x\in A,\\
0, & x\notin A
\end{cases}
$$
is called the characteristic function of the set $A$. 

\end{definition}
\end{purple}

The characteristic function has the following properties:
\begin{enumerate}
    \item The necessary and sufficient condition for $A = X$ is $\chi_A(x)\equiv1$, and the necessary and sufficient condition for $A=\varnothing$ is $\chi_A(x)\equiv0$;
    \item The necessary and sufficient condition for $A\subset B$ is
        $$
        \chi_A(x)\leq\chi_B(x),(\forall x\in X);
        $$
    \item $\chi_{A\cup B}(x)=\chi_A(x)+\chi_B(x)-\chi_{A\cap B}(x)$.
    \item $\chi_{A\cap B}(x)=\chi_A(x)\cdot \chi_B(x)$;
    \item \begin{align*}
        \chi_{\bigcup_{\alpha\in\Lambda}A_{\alpha}}(x) &= \max_{\alpha\in\Lambda}\chi_{A_\alpha}(x)\\
        \chi_{\bigcap_{\alpha\in\Lambda}A_{\alpha}}(x) &= \min_{\alpha\in\Lambda}\chi_{A_\alpha}(x)
    \end{align*}
    \item Let $\{A_k\}$ be an arbitrary sequence of sets, then \begin{align*}
        \chi_{\varlimsup_{k \to \infty} A_k}(x)&=\varlimsup_{k \to \infty}\chi_{A_k}(x)\\
        \chi_{\underline{\lim}_{k \to \infty} A_k}(x)&=\varliminf_{k \to \infty}\chi_{A_k}(x);\\
    \end{align*}
    \item The necessary and sufficient condition for $\lim_{k \to \infty} A_k$ to exist is that $\lim_{k \to \infty} \chi_{A_k}(x)$ exists $(\forall x \in X)$, and when the limit exists, we have
    $$
    \chi_{\lim_{k \to \infty} A_k}(x)=\lim_{k \to \infty} \chi_{A_k}(x)\quad (x \in X).
    $$
\end{enumerate}

\begin{green}

$$
\varlimsup_{k\to \infty}A_k = \bigcap_{n=1}^{\infty}\bigcup_{k=n}^{\infty}A_k
$$
The limit superior of a sequence of sets reflects the collection of elements that "repeatedly appear" in the sequence of sets.

$$
\varlimsup_{k\to \infty}A_k = \bigcup_{n=1}^{\infty}\bigcap_{k=n}^{\infty}A_k
$$
The limit inferior of a sequence of sets reflects the collection of elements that exhibit "stable membership" within the sequence of sets.

\begin{align*}
\varlimsup_{k\to\infty}\chi_{A_k}(x)&=\lim_{n\to\infty}(\sup_{k\ge n}\chi_{A_k(x)})\\
\varliminf_{k\to\infty}\chi_{A_k}(x)&=\lim_{n\to\infty}(\inf_{k\ge n}\chi_{A_k(x)})
\end{align*}
\end{green}

Next, we will employ the method of characteristic functions to complete the proof of the associativity of the symmetric difference operation.

\begin{proof}
\begin{align*}
    \chi_{A\oplus B}(x)&=\chi_{(A-B)\cup(B-A)}(x)\\
    &=\chi_{A-B}(x)+\chi_{B-A}(x)\\
    &=\chi_{A}(x)\cdot\chi_{\bar B}(x)+\chi_{\bar A}(x)\cdot\chi_{B}(x)\\
    &=\chi_A(x)+\chi_B(x)-2\chi_{A\cap B}(x)\\
    \\
    \text{LEFT}&=\chi_{(U\oplus V)\oplus T}(x)\\
    &=\chi_U(x)+\chi_V(x)+\chi_T(x)-2\chi_{U\cap V}(x)-2\chi_{V\cap T}(x)-2\chi_{U\cap T}(x)+4\chi_{U\cap V \cap T}(x)\\
    &=\text{RIGHT}
\end{align*}
\end{proof}

\begin{purple}
\begin{definition}
\begin{enumerate}
    \item An ordered pair is a collection of sets $\{\{x,y\},\{x\}\}$. It can be readily verified that x and y are determined uniquely.
    \item Let $\{\{x,y\},\{x\}\}$ be an ordered pair. Then $x$ is the first component of $p$ and $y$ is the second component of $p$.
    \item Let $X$, $Y$ be two sets. Their product is the set $X\times Y$ that consists of all pairs of the form $(x, y)$ as an ordered pair on the set $S$.
\end{enumerate}
\end{definition}
\end{purple}


\end{document}


