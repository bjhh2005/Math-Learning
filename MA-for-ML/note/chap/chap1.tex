\documentclass[../main.tex]{subfiles}

\begin{document}

\section{Preliminaries}
\subsection{Set and Collections}
\begin{purple}
\begin{definition}
    Let $\mathcal{D}$ be a collection of sets. The union of $\mathcal{C}$, denoted by $\bigcup \mathcal{C}$, is the set defined by
    $$
    \bigcup\mathcal{C}=\{x|x\in S \text{ for some } S\in \mathcal{C}\}.
    $$

    If $\mathcal{C}$ is a non-empty collection, its intersection is the set $\bigcap\mathcal{C}$ given by 
    $$
    \bigcap\mathcal{C} = \{x|x\in S \text{ for every } S \in \mathcal{C}\}.
    $$
\end{definition}
\end{purple}

\begin{green}
Here, the generalized union and generalized intersection of sets simply involve performing the union or intersection operation on each element within the set.
\end{green}

Note that if $\mathcal{C}$ and $\mathcal{D}$ are two collections such that $\mathcal{C}\subseteq\mathcal{D}$, then
$$
\bigcup \mathcal{C}\subseteq \bigcup\mathcal{D} \text{ and }
\bigcap \mathcal{D} \subseteq \bigcap \mathcal{C}
$$

Within the framework of collections of subsets of a given set $\mathcal{S}$, we extend the previous definition by taking $\bigcap \varnothing = S$ for the empty collection of subsets of $\mathcal{S}$.

\begin{green}
    This is consistent with the fact that $\varnothing \subseteq \mathcal{C}$ implies $\bigcap \mathcal{C} \subseteq \mathcal{S}$. 
\end{green}

\begin{purple}
\begin{definition}
    The symmetric difference of sets denoted by $\oplus$ is defined by $U \oplus V = (U - V) \cup (V - U)$ for all sets $U, V$.
\end{definition}
\end{purple}

The symmetric difference operation is easily shown to satisfy symmetry and associativity, and also $U\oplus U=\varnothing$.

Next, we will prove the associativity.

\begin{proof}

    \begin{align*}
    \text{LEFT} 
    &= ((U - V) \cup (V - U)) \oplus T\\
    &= (((U - V) \cup (V - U)) - T) \cup (T - ((U - V) \cup (V - U)))\\
    &= ((U\bar V \cup V\bar U)\cap \bar T) \cup (T\cap (\bar U\bar V \cup UV))\\
    &= \sum_{i=1,2,4,7}m_i\\
    \\
    \text{RIGHT}
    &=\text{LEFT}\\
\end{align*}

\end{proof}

However, the above method is rather cumbersome. We can instead adopt the characteristic function approach for the proof. Here, we first supplement the relevant concepts of characteristic functions.

\begin{purple}
\begin{definition}
For any set $A$, the function
$$
\chi_A(x)=\begin{cases}
1, & x\in A,\\
0, & x\notin A
\end{cases}
$$
is called the characteristic function of the set $A$. 

\end{definition}
\end{purple}

The characteristic function has the following properties:
\begin{enumerate}
    \item The necessary and sufficient condition for $A = X$ is $\chi_A(x)\equiv1$, and the necessary and sufficient condition for $A=\varnothing$ is $\chi_A(x)\equiv0$;
    \item The necessary and sufficient condition for $A\subset B$ is
        $$
        \chi_A(x)\leq\chi_B(x),(\forall x\in X);
        $$

    \item $\chi_{\bar P}(x) = 1 - \chi_{P}(x)$.
    \item $\chi_{A\cup B}(x)=\chi_A(x)+\chi_B(x)-\chi_{A\cap B}(x)$.
    \item $\chi_{A\cap B}(x)=\chi_A(x)\cdot \chi_B(x)$;
    \item \begin{align*}
        \chi_{\bigcup_{\alpha\in\Lambda}A_{\alpha}}(x) &= \max_{\alpha\in\Lambda}\chi_{A_\alpha}(x)\\
        \chi_{\bigcap_{\alpha\in\Lambda}A_{\alpha}}(x) &= \min_{\alpha\in\Lambda}\chi_{A_\alpha}(x)
    \end{align*}
    \item Let $\{A_k\}$ be an arbitrary sequence of sets, then \begin{align*}
        \chi_{\varlimsup_{k \to \infty} A_k}(x)&=\varlimsup_{k \to \infty}\chi_{A_k}(x)\\
        \chi_{\underline{\lim}_{k \to \infty} A_k}(x)&=\varliminf_{k \to \infty}\chi_{A_k}(x);\\
    \end{align*}
    \item The necessary and sufficient condition for $\lim_{k \to \infty} A_k$ to exist is that $\lim_{k \to \infty} \chi_{A_k}(x)$ exists $(\forall x \in X)$, and when the limit exists, we have
    $$
    \chi_{\lim_{k \to \infty} A_k}(x)=\lim_{k \to \infty} \chi_{A_k}(x)\quad (x \in X).
    $$
\end{enumerate}

\begin{green}

$$
\varlimsup_{k\to \infty}A_k = \bigcap_{n=1}^{\infty}\bigcup_{k=n}^{\infty}A_k
$$
The limit superior of a sequence of sets reflects the collection of elements that "repeatedly appear" in the sequence of sets.

$$
\varlimsup_{k\to \infty}A_k = \bigcup_{n=1}^{\infty}\bigcap_{k=n}^{\infty}A_k
$$
The limit inferior of a sequence of sets reflects the collection of elements that exhibit "stable membership" within the sequence of sets.

\begin{align*}
\varlimsup_{k\to\infty}\chi_{A_k}(x)&=\lim_{n\to\infty}(\sup_{k\ge n}\chi_{A_k(x)})\\
\varliminf_{k\to\infty}\chi_{A_k}(x)&=\lim_{n\to\infty}(\inf_{k\ge n}\chi_{A_k(x)})
\end{align*}
\end{green}

Next, we will employ the method of characteristic functions to complete the proof of the associativity of the symmetric difference operation.

\begin{proof}
\begin{align*}
    \chi_{A\oplus B}(x)&=\chi_{(A-B)\cup(B-A)}(x)\\
    &=\chi_{A-B}(x)+\chi_{B-A}(x)\\
    &=\chi_{A}(x)\cdot\chi_{\bar B}(x)+\chi_{\bar A}(x)\cdot\chi_{B}(x)\\
    &=\chi_A(x)+\chi_B(x)-2\chi_{A\cap B}(x)\\
    \\
    \text{LEFT}&=\chi_{(U\oplus V)\oplus T}(x)\\
    &=\chi_U(x)+\chi_V(x)+\chi_T(x)-2\chi_{U\cap V}(x)-2\chi_{V\cap T}(x)-2\chi_{U\cap T}(x)+4\chi_{U\cap V \cap T}(x)\\
    &=\text{RIGHT}
\end{align*}
\end{proof}


\begin{purple}
\begin{definition}
    Below are several concepts pertaining to sets.

\begin{enumerate}
    \item An ordered pair is a collection of sets $\{\{x,y\},\{x\}\}$. It can be readily verified that x and y are determined uniquely.
    \item Let $\{\{x,y\},\{x\}\}$ be an ordered pair. Then $x$ is the first component of $p$ and $y$ is the second component of $p$.
    \item Let $X$, $Y$ be two sets. Their product is the set $X\times Y$ that consists of all pairs of the form $(x, y)$ where $x\in X$ and $y \in Y$.
    \item Let $\mathcal{C}$ and $\mathcal{D}$ be two collections of sets such that $\bigcup \mathcal{C}=\bigcup \mathcal{D}$. $\mathcal{D}$ is a \textbf{refinement} of $\mathcal{C}$ if, for every $D\in\mathcal{D}$, there exists $C\in\mathcal{C}$ such that $D\subseteq C$.This is denoted by $\mathcal{C}\sqsubseteq \mathcal{D}$.
    \item A collection of sets $\mathcal{C}$ is \textbf{hereditary} if $U\in\mathcal{C}$ and $W \subseteq U$ implies $W\in \mathcal{C}$.
    \item The set of subsets of $S$ that contain $k$ elements is denoted by $\mathcal{P}_k(S)$.Clearly, for every set $S$, we have $\mathcal{P}_0(S)=\{\varnothing\}$. The set of all finite subsets of a set $S$ is denoted by $\mathcal{P}_{fin}(S)=\bigcup_{k\in \mathcal{N}}\mathcal{P}_k(S)$.
    \item Let $\mathcal{C}$ be a collection of sets and let $U$ be a set. The \textbf{trace} of the collection $\mathcal{C}$ on the set $U$ is the collection $\mathcal{C}_U=\{U\cap C| C\in \mathcal{C}\}$.
\end{enumerate}
\end{definition}
\end{purple}

\begin{purple}
\begin{definition}

Let $\mathcal{C}$ and $\mathcal{D}$ be two collections of sets.

\begin{enumerate}
    \item $\mathcal{C} \lor \mathcal{D}=\{C\cup D| C \in \mathcal{C}\text{ and } D \in \mathcal{D}\}$,
    \item $\mathcal{C} \land \mathcal{D}=\{C\cap D| C \in \mathcal{C}\text{ and } D \in \mathcal{D}\}$,
    \item $\mathcal{C} - \mathcal{D}=\{C - D| C \in \mathcal{C}\text{ and } D \in \mathcal{D}\}$.
\end{enumerate}
\end{definition}
\end{purple}

Attention, unlike $\cup$ and $\cap$, the operations $\lor$ and $\land$ between collections of sets are not idempotent. Indeed, we have, for example, 
$$
\mathcal{D}\lor\mathcal{D}=\{\{y\},\{x,y\},\{u, y, z\},\{u,x,y,z\}\}\neq\mathcal{D}.
$$

\begin{purple}
\begin{definition}
A \textbf{partition} of a non-empty subsets of $S$ that are pairwise disjoint and whose union equals $S$. 

The \textbf{members} of $\pi$ are referred to as the blocks of the partition $\pi$. The collection of partitions of a set $S$ is denoted by $PART(S)$. A partition is finite if it has a finite number of blocks. The set of finite partitions of $S$ is denoted by $PART_{fin}(S)$.

If $\pi\in PART(S)$ then a subset $T$ of $S$ is $\pi$-saturated if it is a union of blocks of $\pi$.
\end{definition}
\end{purple}

\subsection{Relations and Functions}
\begin{purple}
\begin{definition}
Let $X$, $Y$ be two sets. A relation on $X$, $Y$ is a subset $\rho$ of the set product $X\times Y$. If $X=Y=S$, we refer to $\rho$ as a relation on $S$. The relation $\rho$ on $S$ is: 
\begin{itemize}
    \item reflexive if $(x,x)\in\rho$ for every $x\in S$;
    \item irreflexive if $(x, x) \notin \rho$ for every $x\in S$;
    \item symmetric if$(x,y)\in\rho$ implies $(y,x)\in\rho$ for all $x,y\in S$;
    \item antisymmetric if $(x,y)\in\rho$ and $(y,x)\in\rho$ imply $x=y$ for all $x,y\in S$;
    \item transitive if $(x,y)\in\rho$ and $(y,z)\in\rho$ imply $(x,z)\in \rho$ for all $x, y, z\in S$.
\end{itemize}
\end{definition}
\end{purple}

A \textbf{partial order} on $S$ is a relation $\rho$ that belongs to $\text{REFL}(S)\cap\text{ANTISYMM}(S)\cap\text{TRAN}(S)$, that is, a relation that is reflexive, symmetric and transitive.

In current mathematical practice, we often write $x\rho y$ instead on $(x,y)\in \rho$, where $\rho$ is a relation of $S$ and $x,y\in S$. This alternative way to denote the fact that $(x,y)$ belongs to $\rho$ is known as the infix notation.

\begin{purple}
\begin{definition}
Let $X$, $Y$ be two sets. A function(or a mapping) from $X$ to $Y$ is a relation $f$ on $X$, $Y$ such that $(x, y), (x,y')\in f$ implies $y=y'$.

Let $X$, $Y$ be two sets and let $f:X\rightarrow Y$. 

The domain of $f$ is the set 
$$
\text{Dom}(f)=\{x\in X|y=f(x)\text{ for some } y\in Y\}.
$$
The range of $f$ is the set 
$$
\text{Ran}(f)=\{y\in Y|y=f(x) \text{ for some } x \in X\}.
$$

\end{definition}
\end{purple}

\begin{purple}
\begin{definition}
Let $X$ be a set, $Y = \{0, 1\}$ and let $L$ be a subset of $S$. The characteristic function is discussed above. The indicator function of $L$ is the function $I_L: S\rightarrow \mathcal{R}\cup \infty$ defined by 
$$
I_L(x)=\begin{cases}
    1&\text{if }x\in L.\\
    \infty&\text{otherwise}\\
\end{cases}
$$
for $x\in S$.
\end{definition}
\end{purple}

\begin{purple}
\begin{definition}
A function $f: X\rightarrow Y$ is:
\begin{enumerate}
    \item \textbf{injective or one-to-one} if $f(x_1)=f(x_2)$ implies $x_1=x_2$ for $x_1, x_2\in \text{Dom}(f)$;
    \item \textbf{surjective or onto} if $\text{Ran}(f) = Y$;
    \item \textbf{total} if $\textbf{Dom}(f) = X$.
\end{enumerate}

If $f$ both injective and surjective, then it is a \textbf{bijective} function.
\end{definition}
\end{purple}

\begin{yellow}
\begin{theorem}
A function $f: X\rightarrow Y$ is injective if and only if there exists a function $g: Y\rightarrow X$ such that $g(f(x))=x$ for every $x\in \textbf{Dom}(f)$. 

A function $f: X\rightarrow Y$ is surjective if and only if there exists a function $h: Y\rightarrow X$ such that $f(h(y))=y$ for every $y\in Y$. 
\end{theorem}
\end{yellow}

Here we provide the proof for the latter theorem.

\begin{proof}
Suppose that $f$ is a surjective function. The collection $\{f^{-1}(y)|y\in Y\}$ indexed by $Y$ consists of non-empty sets. By the Axiom of Choice there exists a choice function for this collection, that is a function $h: Y\rightarrow \bigcup_{y\in Y}f^{-1}(y)$ such that $h(y)\in f^{-1}(y)$, or $f(h(y))=y$ for $y\in Y$.

Conversely, suppose that there exists a function $h: Y\rightarrow X$ such that $f(h(y))=y$ for every $y\in Y$. Then, $f(x)=y$ for $x=h(y)$, which shows that $f$ is surjective.
\end{proof}

\begin{green}
\textbf{The Axiom of Choice}: Let $\mathcal{C}=\{C_i|i\in I\}$ be a collection of non-empty sets indexed by a set I. There exists a function $\phi: I\rightarrow\bigcup\mathcal{C}$(known as a choice function) such that $\phi(i)\in C_i$ for each $i\in I$.
\end{green}

\begin{yellow}
\begin{theorem}
There is a bijection $\Psi:\mathcal{P}(S)\rightarrow(S\rightarrow\{0,1\})$ between the set of subsets of $S$ and the set of characteristic functions defined on $S$.
\end{theorem}
\end{yellow}

\begin{purple}
\begin{definition}
A set $S$ is indexed be a set $I$ if there exists a surjection $f:I\rightarrow S$. In this case we refer to $I$ as an index set.

If $S$ is indexed by the function $f: I\rightarrow S$ we write the element $f(i)$ just as $s_i$, if there is no risk of confusion.
\end{definition}
\end{purple}

\begin{purple}
\begin{definition}
A sequence of length $n$ on a set $X$ is a function $x:\{0,1,\cdots,n-1\}\rightarrow X$.

At times, we will use the same term to designate a function $x:\{1,2,\cdots,n\}\rightarrow X$.

The set of sequence of length $n$ on a set $X$ is denoted by $\text{Seq}_n(X)$ which is the $n$-fold Cartesian product essentially.

An infinite sequence, or simply, a sequence on a set $X$ is a function $x:\mathcal{N}\rightarrow X$. The set of infinite sequences on $X$ is denoted by $\text{Seq}(X)$.

Let $S_0,\cdots,S_{n-1}$ be $n$ sets. An $n$-tuple on $S_0,\cdots, S_{n-1}$ is a function $t:\{0,\cdots,n-1\}\rightarrow S_0\cup\cdots S_{n-1}$ such that $t(i)\in S_i$ for $0\le i\le n-1$. The $n$-tuple $t$ is denoted by $(t(0),\cdots,t(n-1))$.

The set of all n-tuples on $S_0,\cdots,S_{n-1}$ is referred to as the Cartesian product of $S_0,\cdots, S_{n-1}$ and is denoted by $S_0\times \cdots\times S_{n-1}$.
\end{definition}
\end{purple}


The Cartesian product of a finite number of sets can be generalized for arbitrary families of sets. Let $\mathcal{S}=\{S_i|i\in I\}$ be a collection of sets indexed by the set $I$. The Cartesian product of $\mathcal{S}$ is the set of all functions of the form $s:I\rightarrow \bigcup\mathcal{S}$ such that $s(i)\in S_i$ for every $i\in I$. This set is denoted by $\prod_{i\in I}S_i$.

\begin{purple}
\begin{definition}
Let $\mathcal{S}=\{S_i|i\in I\}$ be a collection of sets indexed by the set $I$. The $j^{th}$ projection (for $j\in I$) is the mapping $p_j:\prod_{i\in I}S_i\rightarrow S_j$ defined by $p_j(s)=s(j)$ for $j\in I$.

Let $X$, $Y$ be two sets and let $f: X\rightarrow Y$ be a function. If $U\subseteq\text{Dom}(f)$, the image of $U$ under $f$ is the set 
$$
f(U)=\{y\in Y|y=f(u)\text{ for some } u\in X\}.
$$
If $T\subseteq Y$, the pre-image of $T$ under $f$ is the set $f^{-1}(T)=\{x\in X|f(x)\in Y\}$.
\end{definition}
\end{purple}

\begin{yellow}
\begin{theorem}
Let $X$,$Y$ be two sets and let $f:X\rightarrow Y$ be a function. If $V\subseteq Y$, then $X-f^{-1}(V)=f^{-1}(Y-V)$.
\end{theorem}
\end{yellow}

\begin{green}
    
\textbf{Theorem Statement: } The preimage of the complement of $V$ is equal to the complement of the preimage of $V$.

\end{green}


\begin{proof}
    
Let $x \in X - f^{-1}(V)$. Then $x \notin f^{-1}(V)$, so $f(x) \notin V$. Since $f(x) \in Y$, it follows that $f(x) \in Y - V$. By the definition of preimage, $x \in f^{-1}(Y - V)$. Thus, $X - f^{-1}(V) \subseteq f^{-1}(Y - V)$.

Let $x \in f^{-1}(Y - V)$. Then $f(x) \in Y - V$, which implies $f(x) \notin V$. Therefore, $x \notin f^{-1}(V)$, and since $x \in X$, we have $x \in X - f^{-1}(V)$. Hence, $f^{-1}(Y - V) \subseteq X - f^{-1}(V)$.

Since both inclusions hold, we conclude that $X - f^{-1}(V) = f^{-1}(Y - V)$.

\end{proof}

\begin{yellow}
\begin{theorem}
Let $f:X\rightarrow Y$ be a function. We have $U\subseteq f^{-1}(f(U))$ for every subset $U$ of $X$ and $f(f^{-1}(V))\subseteq V $ for every subset $V$ of $Y$.

Let $f: X\rightarrow Y$ be a function. If $U$, $V\subseteq\text{Dom}(f)$ we have $f(U\cup V)=f(U)\cup f(V)$ and $f(U\cap V)\subseteq f(U)\cap f(V)$.

If $S$,$T\subseteq\text{Ran}(f)$, then $f^{-1}(S\cup T)=f^{-1}(S)\cup f^{-1}(T)$ and $f^{-1}(S\cap T)=f^{-1}(S)\cap f^{-1}(T)$.
\end{theorem}
\end{yellow}


\textbf{1. Why does the image "shrink"?}

\textbf{Reason: The function may be a ``partial mapping'' or ``not surjective.''}
\begin{itemize}
    \item Given $f: X \to Y$ and $V \subseteq Y$, when computing $f(f^{-1}(V))$:
    \begin{itemize}
        \item $f^{-1}(V)$ takes all elements in $X$ that can be mapped to $V$.
        \item However, $f$ may not cover the entire $Y$ (i.e., $f$ is not necessarily surjective), so $f(f^{-1}(V))$ only contains the part of $V$ that is \textit{actually} mapped by $f$. Thus:
        $$
            f(f^{-1}(V)) \subseteq V
        $$
    \end{itemize}
    \item \textbf{Extreme cases:}
    \begin{itemize}
        \item If $V$ is entirely outside the image of $f$ (i.e., $V \cap f(X) = \emptyset$), then $f^{-1}(V) = \emptyset$, so $f(f^{-1}(V)) = \emptyset \subseteq V$ (still holds but shrinks to the empty set).
        \item If $f$ is surjective, then $f(f^{-1}(V)) = V$ (equality holds).
    \end{itemize}
\end{itemize}

\textbf{2. Why does the preimage "expand"?}

\textbf{Reason: The function may be ``many-to-one".}
\begin{itemize}
    \item Given $U \subseteq X$, when calculating $f^{-1}(f(U))$:
    \begin{itemize}
        \item $f(U)$ is the image of $U$ in $Y$.
        \item $f^{-1}(f(U))$ takes all elements in $X$ that can be mapped to $f(U)$.
        \item Since $f$ may be ``many-to-one", there may be additional $x \notin U$ that are also mapped to $f(U)$, so:
        $$
            U \subseteq f^{-1}(f(U))
        $$
    \end{itemize}
    \item \textbf{Extreme cases:}
    \begin{itemize}
        \item If $f$ is injective (one-to-one), then $U = f^{-1}(f(U))$ (the equality holds).
        \item If $f$ is a constant function (all $x$ are mapped to the same $y$), then $f^{-1}(f(U)) = X$ (great expansion).
    \end{itemize}
\end{itemize}

\begin{yellow}
\begin{theorem}
    
Let $X$, $Y$ be two sets and let $f:X\rightarrow Y$ be a function. If $U\subseteq X$, then $\text{Ran}(f)-f(U)\subseteq f(\text{Dom}(f) - U)$. If $f$ is injective, then $\text{Ran}(f)-f(U)=f(\text{Dom}(f)-U)$. 

Furthermore, let $f:X\rightarrow X$ be an injection. We have $f(\overline U)=\overline{f(U)}$ for every subset $U$ of $X$.

\end{theorem}
\end{yellow}

\begin{green}
    
\textbf{Theorem Statement:} All images not mapped by $U$ must come from the mapping of $X-U$. If $f$ is injective, then the two sets are exactly equal.

\end{green}


\begin{yellow}
\begin{theorem}

Let $f:X\rightarrow Y$ be a function. If $V,W\subseteq Y$, then $f^{-1}(V-W)=f^{-1}(V)-f^{-1}(W)$. Furthermore, if $\mathcal{C}=\{Y_i|i\in I\}$ is a collection of subsets of $Y$ we have $f^{-1}(\bigcap_{i\in I}C_i)=\bigcap_{i\in I}f^{-1}(C_i)$ and $f^{-1}(\bigcup_{i\in I}C_i)=\bigcup_{i\in I}f^{-1}(C_i)$.

\end{theorem}
\end{yellow}


\begin{proof}
We prove each part separately.

\noindent\textbf{Part 1:} $f^{-1}(V-W)=f^{-1}(V)-f^{-1}(W)$

($\subseteq$) Let $x \in f^{-1}(V-W)$. Then $f(x) \in V-W$, so $f(x) \in V$ and $f(x) \notin W$. 
Therefore, $x \in f^{-1}(V)$ and $x \notin f^{-1}(W)$, which implies $x \in f^{-1}(V)-f^{-1}(W)$. 
Thus, $f^{-1}(V-W) \subseteq f^{-1}(V)-f^{-1}(W)$.

($\supseteq$) Conversely, let $x \in f^{-1}(V)-f^{-1}(W)$. Then $x \in f^{-1}(V)$ and $x \notin f^{-1}(W)$, 
so $f(x) \in V$ and $f(x) \notin W$. Hence, $f(x) \in V-W$, which means $x \in f^{-1}(V-W)$. 
Thus, $f^{-1}(V)-f^{-1}(W) \subseteq f^{-1}(V-W)$.

Therefore, $f^{-1}(V-W) = f^{-1}(V)-f^{-1}(W)$.

\medskip

\noindent\textbf{Part 2:} $f^{-1}\left(\bigcap_{i\in I}Y_i\right) = \bigcap_{i\in I}f^{-1}(Y_i)$

($\subseteq$) Let $x \in f^{-1}\left(\bigcap_{i\in I}Y_i\right)$. Then $f(x) \in \bigcap_{i\in I}Y_i$, 
so for every $i \in I$, $f(x) \in Y_i$. This implies that for every $i \in I$, $x \in f^{-1}(Y_i)$. 
Thus, $x \in \bigcap_{i\in I}f^{-1}(Y_i)$. Hence, $f^{-1}\left(\bigcap_{i\in I}Y_i\right) \subseteq \bigcap_{i\in I}f^{-1}(Y_i)$.

($\supseteq$) Conversely, let $x \in \bigcap_{i\in I}f^{-1}(Y_i)$. Then for every $i \in I$, $x \in f^{-1}(Y_i)$, 
so for every $i \in I$, $f(x) \in Y_i$. This means $f(x) \in \bigcap_{i\in I}Y_i$, 
so $x \in f^{-1}\left(\bigcap_{i\in I}Y_i\right)$. 
Thus, $\bigcap_{i\in I}f^{-1}(Y_i) \subseteq f^{-1}\left(\bigcap_{i\in I}Y_i\right)$.

Therefore, $f^{-1}\left(\bigcap_{i\in I}Y_i\right) = \bigcap_{i\in I}f^{-1}(Y_i)$.

\medskip

\noindent\textbf{Part 3:} $f^{-1}\left(\bigcup_{i\in I}Y_i\right) = \bigcup_{i\in I}f^{-1}(Y_i)$

($\subseteq$) Let $x \in f^{-1}\left(\bigcup_{i\in I}Y_i\right)$. Then $f(x) \in \bigcup_{i\in I}Y_i$, 
so there exists some $i \in I$ such that $f(x) \in Y_i$. This implies that $x \in f^{-1}(Y_i)$ for that $i$, 
so $x \in \bigcup_{i\in I}f^{-1}(Y_i)$. 
Hence, $f^{-1}\left(\bigcup_{i\in I}Y_i\right) \subseteq \bigcup_{i\in I}f^{-1}(Y_i)$.

($\supseteq$) Conversely, let $x \in \bigcup_{i\in I}f^{-1}(Y_i)$. Then there exists some $i \in I$ such that $x \in f^{-1}(Y_i)$, 
so $f(x) \in Y_i$. Since $Y_i \subseteq \bigcup_{i\in I}Y_i$, we have $f(x) \in \bigcup_{i\in I}Y_i$, 
which means $x \in f^{-1}\left(\bigcup_{i\in I}Y_i\right)$. 
Thus, $\bigcup_{i\in I}f^{-1}(Y_i) \subseteq f^{-1}\left(\bigcup_{i\in I}Y_i\right)$.

Therefore, $f^{-1}\left(\bigcup_{i\in I}Y_i\right) = \bigcup_{i\in I}f^{-1}(Y_i)$.

This completes the proof of all three parts.
\end{proof}

\begin{green}

The inverse image operation $f^{-1}$ completely preserves all basic set operations:
\begin{itemize}
    \item Difference set: $f^{-1}(V - W) = f^{-1}(V) - f^{-1}(W)$
    \item Arbitrary intersection: $f^{-1}\left(\bigcap_{i\in I} C_{i}\right) = \bigcap_{i\in I} f^{-1}(C_{i})$
    \item Arbitrary union: $f^{-1}\left(\bigcup_{i\in I} C_{i}\right) = \bigcup_{i\in I} f^{-1}(C_{i})$
\end{itemize}

\end{green}

\begin{purple}
\begin{definition}

Let $X$, $Y$ be two finite non-empty and disjoint sets and let $\rho$ be a relation, $\rho\subseteq X\times Y$. A perfect matching for $\rho$ is an injective mapping $f:X\rightarrow Y$ such that if $y=f(x)$, then $(x,y)\in \rho$.

For a subset $A$ of $X$ define the set $\rho[A]$ as 
$$
\rho[A]=\{y\in Y|(x,y)\in\rho\text{ for some} x\in A\}.
$$
\end{definition}
\end{purple}

\begin{yellow}
\begin{theorem}

\textbf{(Hall's Perfect Matching Theorem)}: Let $X$,$Y$ be two finite non-empty and disjoint sets and let $\rho$ be a relation, $\rho\in X\times Y$. There exists a perfect matching for $\rho$ if and only if for every $A\in \mathcal{P}(X)$ we have $|\rho[A]|\ge|A|$. 
\end{theorem}
\end{yellow}

\begin{proof}
The proof is by induction on $|X|$. If $|X|=1$, the statement is immediate.

Suppose that the statement holds for $|X|\leqslant n$ and consider a set $X$ with $|X|=n+1$. We need to consider two cases: either $|\rho[A]|>|A|$ for every subset $A$ of $X$, or there exists a subset $A$ of $X$ such that $|\rho[A]|=|A|$.

In the first case, since $|\rho[\{x\}]|>1$ there exists $y\in Y$ such that $(x,y)\in\rho$.

Let $X^{\prime}=X-\{x\}$, $Y^{\prime}=Y-\{y\}$, and let $\rho^{\prime}=\rho\cap(X^{\prime}\times Y^{\prime})$. Note that for every $B\subseteq X^{\prime}$ we have $|\rho^{\prime}[B]|\geqslant|B|+1$ because for every subset $A$ of $X$ we have $|\rho[A]|\geqslant|A|+1$ and deleting a single element $y$ from $\rho[A]$ still leaves at least $|A|$ elements in this set. By the inductive hypothesis, there exists a perfect matching $f^{\prime}$ for $\rho^{\prime}$. This matching extends to a matching $f$ for $\rho$ by defining $f(x)=y$.

In the second case, let $A$ be a proper subset of $X$ such that $|\rho[A]|=|A|$. Define the sets $X^{\prime},Y^{\prime},X^{\prime\prime},Y^{\prime\prime}$ as
$$
\begin{array}{ll}
X^{\prime}=A, & X^{\prime\prime}=X-A, \\
Y^{\prime}=\rho[A], & Y^{\prime\prime}=Y-\rho[A]
\end{array}
$$
and consider the relations $\rho^{\prime}=\rho\cap(X^{\prime}\times Y^{\prime})$, and $\rho^{\prime\prime}=\rho\cap(X^{\prime\prime}\times Y^{\prime\prime})$. We shall prove that there are perfect matchings $f^{\prime}$ and $f^{\prime\prime}$ for the relations $\rho^{\prime}$ and $\rho^{\prime\prime}$. A perfect matching for $\rho$ will be given by $f^{\prime}\cup f^{\prime\prime}$.

Since $A$ is a proper subset of $X$ we have both $|A|\leqslant n$ and $|X-A|\leqslant n$.

For any subset $B$ of $A$ we have $\rho^{\prime}[B]=\rho[B]$, so $\rho^{\prime}$ satisfies the condition of the theorem and a perfect matching $f^{\prime}$ for $\rho^{\prime}$ exists.

Suppose that there exists $C\subseteq X^{\prime\prime}$ such that $|\rho^{\prime\prime}[C]|<|C|$. This would imply $|\rho[C\cup A]|<|C\cup A|$ because $\rho[C\cup A]=\rho^{\prime\prime}[C]\cup\rho[A]$, which is impossible. Thus, $\rho^{\prime\prime}$ also satisfies the condition of the theorem and a perfect matching exists for $\rho^{\prime\prime}$.
\end{proof}

\begin{green}
    
\textbf{Implication:} The theorem reduces the problem of matching existence to \textbf{verifying local constraints on every subset of the graph}, bypassing the need for explicit matching construction.

\end{green}



\subsection{Sequences and Collections of Sets}

\begin{purple}
\begin{definition}

A sequence of $S=(S_0,S_1,\cdots,S_n,\cdots)$ is expanding if $i<j$ implies $S_i\subseteq S_j$ for every $i,j\in \mathbb{N}$.


If $i<j$ implies $S_j \subseteq S_i$ for every $i,j\in \mathbb{N}$, then we say that $S$ is a contracting sequence of sets.

A sequence of sets is monotone if it is expanding or contracting.
\end{definition}
\end{purple}

\begin{purple}
\begin{definition}
Let $S$ be an infinite sequence of subsets of a set $S$, where $S(i)=S_i$ for $i\in \mathbb{N}$.

The set $\bigcup_{i=0}^\infty\bigcap_{j=i}^{\infty}S_j$ is referred to as the lower limit of $S$; the set $\bigcap_{i=0}^{\infty}\bigcup_{j=i}^{\infty} S_j$ is the upper limit of $S$. These two sets are denoted by $\lim\inf S$ and $\lim\sup S$, respectively.

Clearly, we have $\lim\inf S\subseteq \lim\sup S$.
\end{definition}
\end{purple}

\begin{green}
    
Intuitively, the lower limit contains elements that are \textbf{always present after a certain point}, whereas the upper limit contains elements that are \textbf{repeatedly present, but not necessarily always}.

\end{green}

\begin{purple}
\begin{definition}
    
A sequence of sets $S$ is convergent if $\lim\inf S=\lim\sup S$. In this case the set $L=\lim\inf S=\lim\sup S$ is said to be the limit of the sequence $S$ and is denoted by $\lim S$.
\end{definition}
\end{purple}

\begin{green}
    
Every \textbf{expanding} sequence of sets is convergent.

\end{green}


\begin{purple}
\begin{definition}

Let $\mathcal{C}$ be a collection of subsets of a set $S$. The collection $\mathcal{C}_{\sigma}$ consists of all countable unions of members of $\mathcal{C}$.

The  collection $\mathcal{C}_{\delta}$ consists of all countable intersections of members of $\mathcal{C}$,
$$
\mathcal{C}_{\sigma}=\left\{\bigcup_{n\ge0}C_n|C_n\in \mathcal{C}\right\} \text{ and } \mathcal{C}_{\delta}=\left\{\bigcap_{n\ge0}C_n|C_n\in\mathcal{C}\right\}.
$$
\end{definition}
\end{purple}

\begin{green}
    
Observe that by taking $C_n=C\in\mathcal{C}$ for $n\ge 0$ it follows that $\mathcal{C}\subseteq\mathcal{C}_{\sigma}$ and $\mathcal{C}\subseteq \mathcal{C}_{\delta}$. Furthermore, if $\mathcal{C}$,$\mathcal{C}^\prime$ are two collections of subsets of $S$ and $\mathcal{C}\subseteq \mathcal{C}^\prime$, then $\mathcal{C}_{\sigma}\subseteq \mathcal{C}_{\sigma}^\prime$ and $\mathcal{C}_{\delta}\subseteq \mathcal{C}^\prime_{\delta}$.

\end{green}


\begin{yellow}
\begin{theorem}

For any collection of subsets $\mathcal{C}$ of a set $S$ we have $(\mathcal{C}_\sigma)_{\sigma}=\mathcal{C}_{\sigma}$ and $(\mathcal{C}_{\delta})_{\delta}=\mathcal{C}_\delta$.

\end{theorem}
\end{yellow}

\begin{proof}
We prove both statements separately.

\noindent
\textbf{Part 1: $(\mathcal{C}_\sigma)_\sigma = \mathcal{C}_\sigma$}

\begin{enumerate}
    \item[($\subseteq$)] Let $A \in (\mathcal{C}_\sigma)_\sigma$. Then there exists a countable family $\{A_n\}_{n\in\mathbb{N}} \subseteq \mathcal{C}_\sigma$ such that
    $$
    A = \bigcup_{n\in\mathbb{N}} A_n.
    $$
    For each $A_n$, there exists a countable family $\{C_{n,k}\}_{k\in\mathbb{N}} \subseteq \mathcal{C}$ such that
    $$
    A_n = \bigcup_{k\in\mathbb{N}} C_{n,k}.
    $$
    Therefore,
    $$
    A = \bigcup_{n\in\mathbb{N}} \bigcup_{k\in\mathbb{N}} C_{n,k} = \bigcup_{(n,k)\in\mathbb{N}\times\mathbb{N}} C_{n,k},
    $$
    which is a countable union of elements of $\mathcal{C}$. Hence $A \in \mathcal{C}_\sigma$.

    \item[($\supseteq$)] For any $A \in \mathcal{C}_\sigma$, we can take the trivial union $A = \bigcup_{n\in\mathbb{N}} A_n$ where $A_1 = A$ and $A_n = \emptyset$ for $n \geq 2$. Thus $A \in (\mathcal{C}_\sigma)_\sigma$.
\end{enumerate}

\noindent
\textbf{Part 2: $(\mathcal{C}_\delta)_\delta = \mathcal{C}_\delta$}

\begin{enumerate}
    \item[($\subseteq$)] Let $B \in (\mathcal{C}_\delta)_\delta$. Then there exists a countable family $\{B_n\}_{n\in\mathbb{N}} \subseteq \mathcal{C}_\delta$ such that
    $$
    B = \bigcap_{n\in\mathbb{N}} B_n.
    $$
    For each $B_n$, there exists a countable family $\{C_{n,k}\}_{k\in\mathbb{N}} \subseteq \mathcal{C}$ such that
    $$
    B_n = \bigcap_{k\in\mathbb{N}} C_{n,k}.
    $$
    Therefore,
    $$
    B = \bigcap_{n\in\mathbb{N}} \bigcap_{k\in\mathbb{N}} C_{n,k} = \bigcap_{(n,k)\in\mathbb{N}\times\mathbb{N}} C_{n,k},
    $$
    which is a countable intersection of elements of $\mathcal{C}$. Hence $B \in \mathcal{C}_\delta$.

    \item[($\supseteq$)] For any $B \in \mathcal{C}_\delta$, we can take the trivial intersection $B = \bigcap_{n\in\mathbb{N}} B_n$ where $B_1 = B$ and $B_n = S$ for $n \geq 2$. Thus $B \in (\mathcal{C}_\delta)_\delta$.
\end{enumerate}
\end{proof}

\begin{green}
\textbf{Idempotence Principle}:

Let $\mathcal{C}$ be any collection of sets, with $\sigma$ and $\delta$ operations defined as:
$$
\mathcal{C}_\sigma = \left\{ \bigcup_{n\in\mathbb{N}} C_n \mid C_n \in \mathcal{C} \right\}, \quad 
\mathcal{C}_\delta = \left\{ \bigcap_{n\in\mathbb{N}} C_n \mid C_n \in \mathcal{C} \right\}
$$

The idempotence property holds because \textbf{Closure under countable operations}: A countable union of countable unions remains countable ($\aleph_0 + \aleph_0 = \aleph_0$)

Thus, the idempotence of $\sigma$ and $\delta$ operations is fundamentally a consequence of countability.
\end{green}

The operation $\sigma$ and $\delta$ can be applied iteratively. We denote sequences of applications of these operations by subscripts adorning the affected collection. The order of application coincides with the order of these symbols in the subscript.

\begin{green}
    
Observe that if$C=(C_0,C_1,\cdots)$ is a sequence of sets, then $\lim\sup C=\bigcap_{i=0}^\infty\bigcup_{j=i}^{\infty}C_j\in\mathcal{C}_{\sigma\delta}$ and $\lim\inf C=\bigcup_{i=0}^\infty\bigcap_{j=i}^{\infty}C_j\in\mathcal{C}_{\delta\sigma}$, where $\mathcal{C}$ is the collection of all sets $C_i$ for $i\in\mathbb{N}$.
\end{green}

\subsection{Closure and Interior Systems}

\begin{purple}
\begin{definition}
    
Let $S$ be a set. A closure system on $S$ is a collection $\mathcal{C}$ of subsets of $S$ such that $S\in\mathcal{C}$ and for any family $\{C_i\}_{i\in I}\subseteq \mathcal{C}$, we have $\bigcap_{i\in I}C_i\in\mathcal{C}$.
\end{definition}
\end{purple}

\begin{yellow}
\begin{theorem}

The sets $\text{REFL}(S)$, $\text{STMM}(S)$, $\text{TRAN}(S)$ and $\text{EQUI}(S)$ are closure systems on $S$.
\end{theorem}
\end{yellow}

\begin{purple}
\begin{definition}

A mapping $K:\mathcal{P}(S)\rightarrow\mathcal{P}(S)$ is a closure operator on a set $S$ if it satisfies the conditions:
\begin{itemize}
    \item $U\subseteq K(U)$(expansiveness)
    \item $U\subseteq V\implies K(U)\subseteq K(V)$(monotonicity)
    \item $K(K(U))=K(U)$(idempotency)
\end{itemize}
for $U,V\in\mathcal{P}(S)$.
\end{definition}
\end{purple}

Obviously, $K(S)=S$.

\begin{purple}
\begin{definition}
    Let $S$ be a set. A collection $\mathfrak{M}$ of subsets of $S$ is a \textbf{monotone class} if the following conditions are satisfied:
    \begin{enumerate}
        \item If $C = (C_n)$ is an increasing sequence of sets in $\mathfrak{M}$, then $\bigcup_{n \in \mathbb{N}} C_n \in \mathcal{M}$;
        
        \item If $D = (D_n)$ is a decreasing sequence of sets in $\mathfrak{M}$, then $\bigcap_{n \in \mathbb{N}} D_n \in \mathcal{M}$.
    \end{enumerate}
\end{definition}
\end{purple}

\begin{yellow}
\begin{theorem}

Let $K: \mathcal{P}(S) \to \mathcal{P}(S)$ be a closure operator. Define the family of sets 
$$
\mathcal{C}_K = \{H \in \mathcal{P}(S) \mid H = K(H)\}.
$$ 
Then, $\mathcal{C}_K$ is a closure system on $S$.

\end{theorem}
\end{yellow}

\end{document}


